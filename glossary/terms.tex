
In the specific case of Kanjira/Mridangam, the definitions of different rhythm components are:

\begin{itemize}

\item \textbf{Stroke:} Hit/tap on the instrument's surface.

\item \textbf{Pattern:} Temporal sequence of strokes and pauses played either as solo or as accompaniment to a song. Some examples of patterns are:

\begin{enumerate}

\item "tha. tumki."
\item "tha thum thum tha"
\item "thum tha thum thum tha thum thum tha"

\end{enumerate}

\item \textbf{Diction:} The sequence of strokes in pattern.

\item \textbf{Accents:} The emphasis on certain strokes, emphasis being loudness, increase of pitch, lengthening of note duration.

\item \textbf{Intonation:} Controlling a pitch of the bass sounds.

\item \textbf{Beat:} The smallest unit of rhythm. Counted as beats per minute. 

\item \textbf{Clap:} Set of beats. Denoted by the clap of the hand played periodically after a certain number of beats. If the time signature is 4/4, then 1 clap is played for every 4 beats, if time signature is 5/4, then 1 clap is played for every 5 beats.

\item \textbf{Pause(in a rhythm pattern):} Time interval between the notes played in a rhythm pattern. Pauses refers to the places of silence in a rhythm pattern. Pause is measured in terms of "number of beats". 

\item \textbf{Duration of stroke:} Duration of each stroke played in the pattern. Duration is measured in terms of number of beats. Duration can be used to calculate pauses and vice versa.  

\item \textbf{Phrase:} Refers to a melody phrase or the song.

\end{itemize}
