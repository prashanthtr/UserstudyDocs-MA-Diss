
Rhythm patterns for accompaniment are selected based on 5 characteristic features: Stroke density in a pattern, Duration of pattern, Intonation on resonant strokes, Stroke interval in a pattern and Accent structure.


\emph{Hypothesis about the characteristics of generic accompaniment patterns are}:

\begin{itemize}
\item Stroke density in a pattern - 1, 2 or 4 or 8 beats
\item Duration of pattern (4 or 8 beats) - 4 or 8 beats
\item Intonation on resonant strokes - Y/N
\item Stroke interval in a pattern - 0 or 1 beats
\item Accent structure (1,3 or 5) - 1(accent on beat 1 only)

\end{itemize}

\emph{Hypothesis about characteristics that definitely change between the generic accompaniment patterns and non-generic accompaniment patterns}:

\begin{itemize}

\item Stroke density -- 2,3...8 beats
\item Stroke interval in a pattern - 2,3...8 beats
\item Accent structure - Even numbered note positions(2,4,6,8)

\end{itemize}

\emph{Hypothesis about characteristics that may not change between generic accompaniment patterns and non-generic accompaniment patterns}:

\begin{itemize}

\item Duration of pattern (4 or 8 beats) - 4 or 8 beats
\item Intonation on resonant strokes - Y/N

\end{itemize}

In the first style of accompaniment, generic patterns are played as accompaniment by the system. In the style, the characteristics are varied according to the hypothesis mentioned above, and system plays accompaniment.
