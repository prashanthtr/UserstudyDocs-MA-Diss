;; This buffer is for notes you don't want to save, and for Lisp evaluation.
;; If you want to create a file, visit that file with C-x C-f,
;; then enter the text in that file's own buffer.



give controls for controlling each of the parameter of generic accompaniment

. My hypothesis is that "generic accompaniment" has the following characteristics:

- Accent structure: Usually accents on only the 1st beat or the 1st and 5th beat
- Time interval between strokes: Equal
- Stroke density: 1,2 or 4 strokes per clap(4 beats)
- Diction: Non-resonant strokes are accented. Resonant strokes are non-accented but used to give structure. Non resonant strokes - Tha, thi. Resonant strokes: thum, tumki(Pitch bent version of thum)
- Intonation: Either no intonation or Intonation only on the resonant strokes.

2. Given this, the following are three examples of accompaniment that violate this model: (These are valid accompaniment patterns, however, they violate the conditions required for generic patterns)
a. "tumki. tumki. tha thum thum tha" - Unequal note intervals, resonant strokes are accented
b. 'tha tumki. tha' - Unequal note intervals
c. "thum tha thum thum tha thum thum tha" - Accent on second note

3. Therefore, for my study, I will create and present to participants:
a. [3] accompaniment patterns that I believe are generic
1. "tha. tumki."
2. "tha thum thum tha"
3. "tha tha thum thum tha thum thum tha"
b. [3] accompaniment patterns that violate my model of generic accompaniment ( will be generated by varying the parameters of generic patterns? )


controlling for diction ?? -- yes, same diction as in the generic patterns

two options -- generic, random

time interval -- 1,0
stroke density -- 
duration -- 4,8
intonation -- 0/1
accent structure - 1/3( 1 and 3)/5(1,3 and 5)

generic -- select from a list of patterns
