
\item Brief them about the purpose of the study.

What I am really trying to get down to is a clearer and crisper definition of different accompaniment patterns. Existing definition from  literary text, though is, clear, it is incomplete in the sense that, it does not provide any rigurous justification as to what combinations of actions that a player can do while exploring this space of rhythms are valid/ or not when trying to that satisfy a given musical constraint( does not prevent the flow of the tala). It is left to open interpretation, which is good to boost creativity, but a rigurous model also helps to serve as guideline for learners to follow and adhere to.

Some of the actions that are possible on the rhythms that are played on the tala are -- changing loudness, changing speed of notes, changing time signature, changing the order notes played in the stroke. Combining these actions, the number of rhythms that are possible to generate for an 2 aksharams or 8 beats, is, very very roughly of the order of 4^8.

But, not all the rhythms in this space are meaningful nor interesting when it comes to playing for melody or playing along with lead percussionist. One kind of rhythm, that seem to be acceptable as accompaniment, as mentioned in literary texts on carnatic music drumming, are rhythms that reinfornce the tala structure. In this study, I consider this space of rhythms that reinforce the tala structure for the adi tala(an 8 clap structure). Using the oporations of changing loudness, changing speed of notes, changing time signature and changing the order notes played in the stroke, I generate the space of 4^8 rhythms and present a formal mathematical model using which it is possible to classify accompaniment in terms of accompaniment that reinforces the tala, accompaniment that does not reinforce the tala structure and accompaniment that lies in between.

Over the study, you will listen to 3 different kind of accompaniment patterns played on the kanjira, and evaluate the different kinds of accompaniment based on the questionnarie.



Some of the questions are: Is speed doubling allowed, can the time signature be changed, 

what do you think was the relation of this accompaniment played to the tala 
whether they are reinforce the tala structure, contradict it or no relation,

Which of these accompaniment can be played as a acceptable accompaniment in many in a concert?


Looking at rhythms in this space, the simplest rhythms are those that are accented at the 1st, 5th positions. By accented, I mean, the notes that seem to be more prominent in the pattern. For example, in the mridangam, "num" is usually an accented note. It is heard much higher in volume compared to the rest. 


