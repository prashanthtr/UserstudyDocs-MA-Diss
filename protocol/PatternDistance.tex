;; This buffer is for notes you don't want to save, and for Lisp evaluation.
;; If you want to create a file, visit that file with C-x C-f,
;; then enter the text in that file's own buffer.


have a rough idea of what is generic -- symmetry, accent structure, tala structure
have a mathematical model that has boundary limits using which patterns can be classified as generic and non generic
0 - 0.5  generic
0.5 - 1.75 generic and non generic boundary
1.75+ - non generic

But which operation takes it how far away from genericcity -- 

The distance measure between the pattern and the tala is measured as a function of:
Accent structure  A: [0 0 1 0 1 1 0 0]
Tala:             T: [1 0 0 0 1 0 0 0]
Loudness sequence L: [s w w w s w s w] {Sum of Loudness + weight of individual beats}
Speed Sequence    S: [1 1 1 1 [2,2] 1 1 1 1]

Distance = Sum( (A[i] - T[i])* { weight1*(L[i] - T[i]) + weight2*(S[i] - T[i]) } ),

In the above equation, speed and loudness have an effect on the distance only if they contribute to an accent structure at the beat. 
Using the above equation, it is possible to find how far is a pattern from the tala. Patterns whose distance is 0, are close to the tala and do not
obstruct the tala, there are patterns that are within 0-1.75 and >2 units of distance away from the tala. 

What I am trying to find is, whether the patterns that are generic can be played in many situations in a concert ?

There are patterns that have the properties of generic. The problem is that there is no formal model as to how to
play these patterns, what patterns are correct and what are not.
However, there are strong conventions as to certain patterns sound better in a situation and certain others
are totally wrong. This creates a space of patterns that can be explored by percusssionists.
What I have is a method to explore the space of patterns, using some of the actions the percussionists do: speed double, increase volume, stroke seq
Any pattern, can be broken into a combination of these 3 factors and a combination of weights assigned to these is used to get the accent structure of the pattern
The accent structure of the pattern is then compared to the tala, which is the baseline of rhythmic accompaniment in carnatic music.
Certain patterns reinforce the accent structure of tala and certain patterns dont.
What I am really trying to find out are patterns that reinforce the accent structure of the tala.
using this, patterns that sound close to the tala both by the ear and mathematically can be generated.
Now, it is possible to generate patterns that are mathematically away from the tala and see what are the situations
that they can be used in.

What are the variations of the tala that are acceptable as generic accompaniment
Are patterns that are close to the tala/reinforce the tala as per the distance measure, generic accompaniment? -- which means do they answer all the properties  of generic accompaniment that I have mentioned above?
Is there some point after which percussionists no longer think that the variations have property of generic accompaniment?
Is that a mathematical point of departure between generic/ boundary and non generic?



one thing I can try to do is to test these patterns.
what I'm also trying to fix is the boundary values between generic and non generic. 

Using this, it is then possible to reason out why certain patterns are generic and why certain others
are non generic.

Generate a pattern, loudness array, speed array
Parse out the notes on the pattern
get the weights due to individual notes in the pattern
get the weights due to 2 played in the position of 1 beat
get the accent structure
