;; This buffer is for notes you don't want to save, and for Lisp evaluation.
;; If you want to create a file, visit that file with C-x C-f,
;; then enter the text in that file's own buffer.

generic accompaniment is accompaniment that can be played as fall back rhythm pattern in a concer, requires very less effort(in terms of playing), minimum prediction on the part of the percusionist( minimum attention in terms of nuances in the rhythm played), follows the structure of the tala, requires minimum dynamics on the part of the percussionist, but still can be played as accompaniment that suits the context, that does not interfere with the context but it is not necessarily the greatest/best accompaniment that will be said as the most excellent accompaniment suited to the context, but it can be played as accompaniment, that people will not think as musically meaningless when played in a context.

Generic accompaniment is that accompaniment that can be played in most situations. One property of generic accompaniment is that it does not contradict tala structure. Another property is that these rhythms do not interfere with what other players play. What I am able to generate mathematically are rhythms that do not contradict/ contradict tala structure. What I am trying to find out whether the rhythms generated by my mathematical model are rhythms that can be played as generic accompaniment.

Protocol Design:

No of sections in the study:

There were 2 sections in the study. In the first section, the participants listened and evaluated the accompaniment with only the song and in the second section, participants, listened and evalauted the accompaniment with song and with lead improvisation. The task of the participants in each section of the study was to listen to 3 different kinds of accompaniment played by the system and answer whether the accompaniment was generic accompaniment that could be played as generic accompaniment. 


The questionnarire containts all the properties of generic accompaniment that I think are crucial. based on which property is violated, it is possible to incorporate that in the refined definition of genericcity.

relation of the accompaniment to the tala structure. The relation was in terms of whether it reinforced tala structure or whether it was complementary or whether it had no relation to tala structure.

Task1:

Only kanjira accompaniment to melody. The three kinds of accompaniment are 


Some of the rhythms generated are generic accomapniment
some of the rhythms are not -- refine the model ?
why are those rhythms not generic? What situations do they not become generic?
are they generic under certain situations or always non generic?


use graphs to show the relation of variation of accent structure with diction, loudness and speed doubling

accent structure is an indicator of what beats the notes are played on,
and 

what is finally refined in the model are its rules -- say
whatever data u need to get is the data that u need to test the rules

2 levels of rules

1st level is the weight assignment that is used to get to the accent structure 

second level of rule is that rule system for saying that that certain accent structures reinforce the tala and certain dont'. 

Measure of when an accent structure contradicts another.
  - when the sequence of strong and weak beats do not match
  - correlation measure?

Right now, this is based on a simple distance metric, or by comparing the msd between accent and tala pattern, but is there stronger and rigurous method that really correspnds the ear's intuition about wehther a rhythm is close or away from the 


Things to do:

1) Distance measure that corresponds to ear's intuition about patterns being on or off from  genericcity - related work on complementary rhythms( google drive)

distance measure --


1) accent structure - 0's and 1's - direct binary weighting


2) tempo variations, speed doubling -- negative cost factor 

3) loudness -- positive factor to closeness if it is on 1st and 5th, in all other cases it is a negative factor

4) Alternating hand strokes -- positive contributing factor to closeness
non alternating strokes -- negative contributing factor ( lh, rh strokes)




2) Questionnaire based on the definition of generic accompaniment

generic accompaniment is accompaniment that can be played as fall back rhythm pattern in a concer, requires very less effort(in terms of playing), minimum prediction on the part of the percusionist( minimum attention in terms of nuances in the rhythm played), follows the structure of the tala, requires minimum dynamics on the part of the percussionist, rhythmically simple( no tempo variations, time signature changes) but still can be played as accompaniment that suits the context, that does not interfere with the context but it is not necessarily the greatest/best accompaniment that will be said as the most excellent accompaniment suited to the context, but it can be played as accompaniment, that people will not think as musically meaningless when played in a context.

What I can generate is rhythms that follows the tala structure(in terms of the accent structure) with minimum dynamics( 3 levels of volume dynamics, s, w and 0 ), 1 selected pattern ( that literature references to the be the most commonly played sequence), 

What i can change is 
Range of tempo varitiaions within the pattern 
change the no of selected patterns
provide more volume dynamics
add time signature changes




Code:


distance metric

Stirng insertion as a part of a routine sequence array -- data??
array insertion

reconstruct the array based on parsing it?

all patterns are organized as 
sequence
speed

set of patterns playable in speed 1
set of patterns playable in speed 2
combine and play these, new combinations as training increases
