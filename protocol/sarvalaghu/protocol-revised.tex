
\section{Protocol}

The purpose of the study is to find if the set of beat patterns played by the system are generic accompaniment accompaniment that can be played in most musical situations. The musical context in which the system plays accompaniment is when there are 2 percussionists, a lead percussionist, who plays the \emph{mridangam} and an accompanying second percussionist who plays the \emph{kanjira}. The participants played the role of the lead percussionist and system played as the second percussionist. Before starting the study, participants were instructed about the constraints of playing with the system. Participants were asked to play only certain kinds of improvisations. The constraints included not leaving rests, no drum rolls, no doubling or quadrupling tempo and no changing of time signature. The performers played improvisations to to a constant tempo given by a metronome or song, in an 8 clap beat cycle with a time-signature of 4/4. 

\subsection{Training}

After the participants were instructed about the system, they went through an initial training with the system. The training consisted of 3 tasks. The first task was free improvisation in which they could play and improvise any rhythm pattern. This was done to familiarize them with playing on the interface. In the second task, the participants improvised with a metronome at a fixed tempo and played beat patterns. This was done to ensure that they could maintain constant tempo while playing improvisations. In the concluding task of the training, participants played accompaniment to a melodic phrase in order to train them to play accompaniment to melody. The participants went through multiple training trails till they were convinced that they could play lead accompaniment to melody without any noticeable timing errors. Once they were ready to start, they were informed about the details of the actual study. 


\subsection{Task Descriptions}

Participants did 3 tasks in the study. In the first task, they played lead improvisations to melodic phrases. In the second task, they played specific improvisations as instructed. In the third task, they listened and evaluated the accompaniment played by the system to the lead improvisation played by another lead player. In each task, the system played 2 styles of accompaniment. In the first style, the system generated the accompaniment patterns based on a set of conditions of generic patterns. In the second style, system generated accompaniment patterns that violated the set of conditions of generic patterns. After each task, the participants compared and evaluated both the styles and gave their response in the form of a questionnaire. Participants were also allowed to repeat any task if they were found to make mistakes while playing or if they felt that repeating the task was essential to help them evaluate the accompaniment. 


\subsection{Study}

Before looking at the different tasks of the study in detail, let us look at the set of conditions for generic and non-generic patterns. For generic patterns, the conditions are:

\begin{itemize}

\item No of strokes in a pattern: 1, 2, 4 and 8

\item Accent structure of the pattern:
Patterns that have a length of 4 beats have accent only on the 1st beat. Patterns that have a length of 8 beats, have accent on the 2nd, 5th and the 8th beats. 

\item Interval between notes in a pattern:

\end{itemize}

Generic patterns of 4 beat duration have a equal note interval between the different notes in the pattern. The interval between the notes was kept at 1 or 0 beats. For 8 beat patterns, the interval between notes was either kept at 0 or 1 or the pattern followed a 2-3-3 or accent structure.

The above mentioned values of these parameters were used to generate accompaniment that fits the criteria for generic patterns. For playing accompaniment patterns that violates the conditions for generic patterns, one of these parameters is varied keeping the others constant. Finally, two other parameters were kept constant for both the styles, which are the duration of patterns and the presence or absence of intonation of the beats. 


\subsubsection{Task 1}

In the first task, participants played lead improvisations with melody, under the specified musical constraints imposed by the system. The performers were presented with 2 different melodic phrases to which they play lead improvisations. Before starting to play, they listened to the phrases for 2 times in order to familiarize themselves with the melody and form an idea of what lead improvisation could be played. Then, they played the lead improvisations for 4 times for each melodic phrase. For the first time, system generated accompaniment using the settings of generic patterns and for the next 3 times, the system varied the no of strokes, accent structure and interval between the notes, one at a time, using a random number generator, and generated accompaniment that violates the condition for generic patterns. It is important to note that though participants played for four times, they were told that the system plays only 2 styles of accompaniment and that the accompaniment played during the last 3 times were of the same style. After playing with the system for both the melodic phrases, the participants were asked to distinguish, evaluate and compare the two styles of accompaniment played by the system. 

While playing accompaniment, the system waits for a specific duration before starting to play accompaniment. This was done because, in Carnatic music, lead percussionist first starts playing the improvisation and secondary percussionist joins the improvisation after a certain number of beats. The number of beats that system delays before playing accompaniment was set to either 8 or 12 claps in each trial. While conducting the pilot studies, it was found that the entry of the system into the performance was abrupt when it started playing accompaniment patterns. Hence, in order to account for this and help performer's evaluation, the accompaniment played by the system for the first 8 claps were fixed beforehand for both the styles. 

\subsubsection{Task 2}

In the second task, participants were restricted to play only 3 patterns for accompaniment. The three patterns were:

\begin{itemize}
\item "num. thin."
\item "num thin thin thin"
\item "num tha thin thin thi thin thin thi"
\end{itemize}

The three patterns were selected so that they were either the same or they matched the generic patterns played by the system. The syllables used to represent these pattern are different since these are played on the lead instrument. The same patterns are used since they would sound good when the system plays the same patterns along with the lead. The same 2 melodic phrases were used for this task. After playing accompaniment for each melodic phrases in both the accompaniment styles played by the system, participants were asked to distinguish, evaluate and compare the two styles of accompaniment played by the system. 

\subsubsection{Task 3}

In the final task, performers listened and evaluated the accompaniment played by the system. This followed the same experimental conditions as the previous two, with a difference that the lead percussion was already recorded in the song track. The performer listened to the different styles of accompaniment played by the system for each song and evaluated the accompaniment played by the system. This was in order to account for the possible differences when performers listen and evaluate versus when they play, listen and evaluate. After completing this task, the participants were asked to distinguish, evaluate and compare the two styles of accompaniment played by the system. 




\begin{comment}


what is different about both the accompaniment styles played by the system
accent structure -- The traditional accent structure for generic patterns is duration of 4 beats it is 1,3 and for duration of 8 beats, it is 2,5,8 th beat.
 
no of strokes in a pattern -- change to odd number -- even number and 1 correspond to stroke density of generic patterns. Anything other than violates that genericity 

Stroke interval : there has to be equal stroke interval between the different strokes played in the pattern in case of a generic pattern. Unequal stroke intervals skew the timeliness of the structure and hence cannot be used. 

There were 2 kinds of accompaniment played by the system. The first kind is the generic patterns, with the right variations of the tones. The second kind of patterns violate the generic accompaniment criteria and is generated by varying the 3 parameters mentioned above and generating patterns based on the variation.



> - generic: VERY obvious. means: it is EASY to explain/justify what makes it generic

1. takadimi takajuno

• No of strokes in a pattern: 8
• Accent structure of the pattern: Accented on 1st and 5th notes respectively 
• Rests between notes: None
• Duration : 8
• Intonation: NO

2. nam dhin dhin na dhin dhin na dhin

• No of strokes in a pattern: 8
• Accent structure of the pattern: Accented on 1st, 4th and the 7th notes (3-3-2)
• Rests between notes: None
• Duration : 8
• Intonation: YES

> - generic: "close" to the "boundary" of non-generic

1. takadimi takajuno

• No of strokes in a pattern: 2,4
• Accent structure of the pattern: Accented on 1st and 5th notes respectively 
• Rests between notes: Equal rests (1,4)
• Duration : 8
• Intonation: NO

2. nam dhin dhin na dhin dhin na dhin

• No of strokes in a pattern: 3,7
• Accent structure of the pattern: Accented on 1st, 4th and the 7th notes (3-3-2)
• Rests between notes: 
• Duration : 8
• Intonation: YES

> - non-generic: VERY obvious. means: it is EASY to explain/justify what makes it non-generic

1. takadimi takajuno

• No of strokes in a pattern: 2,4 (1,3,5,6,7)
• Accent structure of the pattern: Accented on 1st and 5th notes respectively 
• Rests between notes: Unequal rests between notes(Rests based on number of strokes present in pattern)
• Duration : 8
• Intonation: NO

2. nam dhin dhin na dhin dhin na dhin

• No of strokes in a pattern: 3,7
• Accent structure of the pattern: Accented on 1st, 4th and the 7th notes (3-3-2)
• Rests between notes: 
• Duration : 8
• Intonation: YES

> - non-generic: "close" to the "boundary" of generic

1. takadimi takajuno

• No of strokes in a pattern: 2,4
• Accent structure of the pattern: Accented on 1st and 5th notes respectively 
• Rests between notes: Equal rests (1,4)
• Duration : 8
• Intonation: NO

2. nam dhin dhin na dhin dhin na dhin

• No of strokes in a pattern: 3,7
• Accent structure of the pattern: Accented on 1st, 4th and the 7th notes (3-3-2)
• Rests between notes: 
• Duration : 8
• Intonation: YES

\end{comment}
