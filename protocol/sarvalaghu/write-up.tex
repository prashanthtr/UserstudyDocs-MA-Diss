
In this document, some of the characteristics of generic rhythm patterns are identified and methods to vary them are elaborated.

Rhythmic patterns in Carnatic Music can be characterized by 3 components: Diction, beat structure and intonation. Diction is the choice of strokes that form a beat pattern. Beat structure is the emphasis on certain strokes that gives a recognizable shape to the beat pattern. Intonation comprises of all the other things that performers do, like volume dynamics, controlling the bass or treble, tone of instrument etc. 

One of the main conclusions from the recent study is that beat structure plays a more important role than diction or intonation in defining the rhythmic pattern. There are three variables that could be used to that characterize beat structure of rhythm patterns. They are duration of pauses between the notes, density of notes per clap and duration of the pattern. In terms of finding these characteristics for generic patterns, the second conclusion of the study states the beat structure of generic patterns has to be simple. In terms of simple duration of pauses between notes, the ratio of 1:1 is considered between the notes and pauses. For every note played, a pause is also played immediately. For the simplest note density, 1 or 2 strokes played per clap is considered for generic patterns. If a single note is played, it is played exactly when the clap is played and if two 2 notes were played, one note is played on the clap and other note is played exactly between 2 claps. For the shortest length of generic pattern, 4 beats is taken as the simplest. 

The beat structure of the patterns could be varied along either of note density, pauses between the claps or duration of the generic patterns. The note density could be increased to a maximum of 4 notes played in every beat. The ratio of the pauses between the notes could be varied to 1:2, 1:3 etc. Finally, keeping the other factors constant, the duration of the generic patterns could be increased to 8 or 12 beats.
