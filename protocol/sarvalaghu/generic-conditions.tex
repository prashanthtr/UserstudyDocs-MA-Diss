
\section{Defining Generic accompaniment}

Generic accompaniment is that accompaniment that can be played as reasonably okay accompaniment in most musical situations. One of the main characteristics of this kind of accompaniment is that it should reinforce the beat structure(or the tala). Identifying this in terms of something computable(accents), I have refined the definiton of generic patterns. 

\begin{enumerate}

\item \textbf{Generic:}

Musical goal: Reinforce the beat structure or the flow of the tala ( counted as \textbf{1}234,\textbf{1}234..  )

\begin{itemize}
\item Accents:  Accent only on the 1st, 3rd, 5th and the 7th beats.
\item Symmetry: The accent structure of 1st 4 beats of the pattern is same as the accent structure of the last 4 beats.
\end{itemize}

Examples from Carnatic music: sarvalaghu patterns

\item \textbf{Generic but Boundary:}

Musical goal: Reinforce the beat structure or the flow of the tala but not playable in all situations

\begin{itemize}
\item Accents:  Accent only on the 1st, 3rd, 5th or the 7th beats.
\item Symmetry: No.
\end{itemize}

Examples from Carnatic music: Cross Rhythm patterns, polyrhythm patterns.

\item \textbf{Non generic but on the boundary:}

Musical goal: Contradict the beat structure of tala, but somewhat closer to generic accompaniment

\begin{itemize}
\item Accents:  Accent on any beat.
\item Symmetry: Yes
\end{itemize}

Examples from Carnatic music: Cross Rhythm patterns, polyrhythm patterns.

\item \textbf{Non generic:}

Musical Goal: Contradict the beat structure or flow of the tala, accompaniment that is very specific to the musical context.

\begin{itemize}
\item Accents:  Accent on any beat.
\item Symmetry: No
\end{itemize}

\end{enumerate}

\textbf{ Common across all kinds of the accompaniment patterns:}

\begin{itemize}
\item Duration : 8 beats
\item Diction : Same sequence of strokes
\end{itemize}
