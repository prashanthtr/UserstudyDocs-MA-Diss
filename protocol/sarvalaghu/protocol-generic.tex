;; This buffer is for notes you don't want to save, and for Lisp evaluation.
;; If you want to create a file, visit that file with C-x C-f,
;; then enter the text in that file's own buffer.



Model of generic accompaniment:

Generic: Accenting such that same pattern repeats when the accents are reversed 

Generic on the boundary: Same as generic, but increased number of rolls. Less generic because, the rolls can played only when lead percussionist starts playing the rolls.

Non Generic on the boundary: Pattern is not same as its reverse, number of rolls same as generic. 

Non generic: Highly tailored accompaniment, highly reactive to the lead percussionist in terms of intermixing generic patterns and rolls.

A single pattern, that is often played in concerts(from transcriptions) was selected and four variations of the pattern were produced. This pattern was takadimi takajono. 

In the first variation, generic patterns were created out of the pattern. generic patterns are the same when they are reversed.

In the second variation, the number strokes, in the generic space, that were speed doubled were increased to >2.  

In the third variation, number of strokes was less than 2, however, the accompaniment violated the condition of being same as its reverse

Highly tailored accompaniment selected from a transcription.



Model of generic accompaniment:

Generic: accenting such that same pattern repeats when inversed (No accenting, accenting on 1 & 5 surely, maybe on 3 and/or 7, )

Generic on the boundary: Same as generic, but increased number of rolls, feels more reactive to the lead percussionist/ melodic

Non Generic on the boundary: 
    accenting on 1 & 5 surely, maybe on 3 and/or 7 and maybe on other beats also( not very clear)
    pattern is not the inverse of itself but repeats itself at 2/6 itself, contradicting meter

Non generic: Highly tailored accompaniment, highly reactive to the lead percussionist in terms of intermixing rolls and farans etc

A single pattern, that is often played in concerts(from transcriptions) was selected and four variations of the pattern were produced. This pattern was takadimi takajono. 

In the first variation, generic patterns were created out of the pattern. generic patterns are the same when they are reversed.

In the second variation, the number strokes, in the generic space, that were speed doubled were increased to >2.  

In the third variation, other kinds of accenting was followed, in which patterns that were reverses of each other were generated
patterns that are not reverses of each other, but changed by 2 or 6
(Nadais have to be more carefully analyzed, but as of now this definiton should do)

Highly tailored accompaniment selected from a transcription

Test, evaluate and validate the model for analyzing/generating generic accompaniment patterns through experts. 

\textbf{crisper definition of generic since it affects the claims that you make}

Patterns that when reversed produce the same pattern too

Generic accompaniment patterns are  those rhythm patterns that could be played as reasonably acceptable accompaniment in most situations. 
Some of the features of generic patterns are that these patters do not obstruct the flow of the tala and highlight the \emph{tala}. Another feature is that these are  usually played by percussionists as a fallback patterns, when the percussionist does not know the song or is not able to predict what the singer is going to sing next{?} and improvise accordingly. One example of this kind of accompaniment which found in carnatic music literature, is that of timekeeping or \emph{sarvalaghu} patterns.

Question: How do they not disturb the melody or what the lead player plays. Usually used as fallback patterns, when the percussionist does not know the song or is not able to predict what the singer is going to sing next{?}.

Non-generic accompaniment are those accompaniment patterns that could be played as improvised variations of generic patterns. These consists of rhythm patterns that break flow of \emph{tala} or patterns that are specifically suited to the beat structure of the melody. One example of this kind of accompaniment is cross rhythms \emph{Nadais} or polyrhythms.


In terms of the different sections of the study, the participants were asked to listen to a song and play sarvalaghu accompaniment to the song. The system played accompaniment as the lead percussionists improvised. The participants were asked to listen to the different kinds of accompaniment played by the system and classify it as generic, generic but on the boundary and non generic. Based on the participants remarks, 



Hypothesis:

Generic patterns should be okay for the percussionists throughout the playing as good accompaniment

Reactions about generic patterns at boundary as patterns that were good at some point but 
interferred with their intentions at other points

Non-generic as highly tailored to the melody and completely contradicting with their intentions of playing


Research questions:

What extent can the different accents of the generic patterns varied to make them stay within the generic space 

To what extent do percussionists tolerate generic patterns to move from generic space to non generic space




Random:

other extreme which is not often not followed in concerts, is to carefully stick to the accent structure of the melody or what the lead percussionist plays, in terms of accents and on strokes/ beat structure of the melody orsynchronize to that


different trade offs: 

When percussionist plays:

Present them with generic patterns that are randomly selected from the model
On the other hand, highly tailored kanjira accompaniment from a recording ( non generic case)
does not really help to validate model right


Present them with generic patterns that are randomly selected from the model
non generic accompaniment 




