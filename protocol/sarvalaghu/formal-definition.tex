;; This buffer is for notes you don't want to save, and for Lisp evaluation.
;; If you want to create a file, visit that file with C-x C-f,
;; then enter the text in that file's own buffer.


Different types of accents

Rhythmic - pauses
melodic accents - 
temporal accents - duration change, grouping position, and event onset
dynamic accents - intensity change.
pitch accents - 

 Interval accents are created when a pitch is substantially higher or lower than those pitches preceding or following it. 

A second type of interval accent results when a tone is preceded by a melodic leap that is larger than surrounding melodic intervals 

A contour accent results from a change in the direction of melodic contour.





Properties of each accent

Accent period in this paper refers to the number of beats separating recurring accents of a given type (m or r). 

Beats are defined as the shortest time spans between tone onsets in
a pattern, for the present purposes (cf. Desain, 1992; Temperley, 2001).

Accent phase refers to the position at which regular cycles of accents begin
in relation to the start of a pattern. In-phase accent structures occur when
the beginning of a pattern initiates a regular cycle of accents, whereas phase
shifts occur when regular cycles start after the pattern begins (which may
imply an anacrusis to the listener)

Between different accents:

Relationships between accent periods for each type are indexed by the ratio of
one accent’s period to the other, called accent period ratio. 

simple (integer) accent period ratio 4:4, such accent period ratios form concordant patterns, whereas discordant patterns result
from noninteger accent period ratios
