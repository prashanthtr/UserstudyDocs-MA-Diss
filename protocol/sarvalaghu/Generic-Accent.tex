
\section{Analyzing Accompaniment patterns}

In terms of analyzing/generating accompaniment patterns that fit criteria for generic accompaniment, I have identified 3 parameters that are useful. These patterns are used to analyze a rhythm pattern and construct an accent structure. The accent structure is tested for genericcity and is used to also how close/ far away is the pattern from genericcity. The parameters based on which accompaniment patterns are analyzed, are:

\begin{itemize}

\item Loudness(increasing/decreasing) of hits, 
\item Speed doubling certain strokes and Loudness accent
\item Sequence of strokes

\end{itemize}

Each of these contribute a weight to the accent structure. A given pattern is analyzed based on these parameters and their weights are used to compute the accent structure. Lets see how the different parameters weight the accent structure.

\subsection{Loudness of hits}

The loudness of the hits is corresponds to the amplitude level of the hits. The amplitude of the hits could range between 0 to 1. There are 3 amplitude values, ``s, w and 0''.

\begin{itemize}

\item  ``s'' corresponds to the note played by maximum amplitude of 1.0.

\item 0 corresponds to a rest or a pause, in which the note is played with 0 amplitude ( equivalent to not playing the note)

\item ``w'' corresponds to the note played in between maximum and 0 amplitude. The value of ``w'' is currently fixed as 0.5.

\end{itemize}

The result of the loudness accent is a sequence of ``s,w and 0s'', showing the amplitude level of each of the hits in a sequence. 

An example sequence: ``[s w w w s 0 0 w]''

\subsection{Speed doubling}

In speed doubling, 2 notes are played for the duration of 1 beat and this contributes to the accent structure. In speed quadrupling, 4 notes are played for 1 beat.

For example, in the ``1,1,1,1,1,1,1'', each of the 1's scales the tempo at which each stroke of a pattern is played. 
If the 1st and the 5th strokes in the pattern are doubled, the resultant sequence is, ``[2,2],1,1,1,[2,2],1,1,1''. The 2's in the sequence means that the strokes are played at double speed.

\subsection{Diction}

The sequence of hits in the pattern also contributes to the accent structure. The high note ``ta'' has a stronger influence on the accent structure than the bass note ``tum''. The difference in this level also adds to the accent structure. ``ta'' is represented by a ``T'' and ``tum'' is represented by a ``t''.


\section{Constructing the accent Structure}

The accent structure is constructed by adding the weighted value of the different parameters at same stroke positions.

For example, consider the following sequences of loudness, speed doubling and diction.

loudness    : [  1,  0.5, 0.5, 0.5,  1,   0, 0 0.5] \\
Speed double: [[2,2], 1,   1,   1, [2,2], 1, 1, 1 ] \\
Diction     : [  Tt   t    t    T    Tt   t  t  T ] \\
AccentStruct: [  ?    ?    ?    ?    ?    ?  ?  ? ]  \\

Accent structure is computed by the formula:

\textbf{Weighted accent structure[currentStroke]} = \\

 \emph{1* loudness[currentStroke] - 0.25* loudness[previousStroke] - 0.75*loudness[nextStroke]} + 
\emph{0.4 ( if the stroke is speed doubled) +} 
\emph{0.4 (or) 0.2 (depending on whether is stroke is ``Ta'' or ``tum'') }

The weighted accent structure of the above sequence is:  \textbf{[1.1, -0.363, -0.067, -0.637, 1.402, -0.528, -0.335, -0.572]}
 

The binary accent structure is obtained from this by comparing triplets of beats and assigning the accent to the maximum weighted value, is \textbf{[1, 0, 1, 0, 1, 0, 1, 0]}.

This accent structure is compared with genercicity condition to check it fits/violates the condition. The mean square difference between the pattern's accent strucutre and tala's accent structure is used to compute how close/far away is it from the tala.

